\chapter{Usage}

\section{Overview}

To generate G-codes for a project You have to:

\begin{itemize}
\item Load a machine description.
\item Load geometry data.
\item Setup the raw material.
\item Calculate tool-paths.
\item Simulate the result.
\item Export G-codes.
\end{itemize}

\section{Definition of tools in toolbox}

\subsection{General parameters}

Shaft diameter and shaft length describe the part of an tool that
disappears inside the chuck. 

Maximum speed is the max speed, the tool can handle  before it suffers
from a loss of structural integrity.

The feed coefficient $f_z$ is the amount of material a tooth of the tool
can eat away per revolution. 

\begin{table}[htb]
\begin{tabular}{|l|r|r|r|}
\hline
&\multicolumn{3}{|l|}{$f_z$ for a tool diameter of}\\
Material&$d=2\ldots4$ mm&$d=5\ldots8$ mm&$d=9\ldots12$ mm\\
\hline
Aluminum&0.04&0.05&0.10\\
Brass, copper, bronze&0.04&0.05&0.10\\
Steel&0.02&0.03&0.06\\
Thermoplastics&0.05&0.06&0.07\\
\hline
\end{tabular}
\end{table}


I have found no feed coefficient data for different types of wood.

\subsection{Tool-shape}

The shape of the tool is constructed from short segments.
Each segment can either be a cutting or not.
Possible shape elements are:

\begin{itemize}
\item Straight or bend line.
\item Height change followed by a diameter change.
\item Diameter change followed by a height change.
\item Circle, the radius is orthogonal at the top of the segment.
\item Circle, the radius is orthogonal at the bottom of the segment.
\end{itemize}

